\documentclass[11pt]{article}
\usepackage{fullpage}
\usepackage{amsfonts}

\usepackage[utf8]{inputenc}
\usepackage[sfdefault,light]{roboto}
\usepackage[T1]{fontenc}
\usepackage[ngerman]{babel}
\usepackage{amsmath,amssymb}
\usepackage{ntheorem}
\usepackage{mathtools}
\usepackage{calc} 

%Zum einfärben von Texten
\usepackage[usenames,vipsnames,svgnames,table]{xcolor}
%Zum erstellen des Index
\usepackage{makeidx}
\usepackage[colorlinks=false, pdfborder={0 0 0}]{hyperref}
\makeindex
%Tikz 
\usepackage{tikz}
\usetikzlibrary{arrows,automata}
%for \Set
%\usepackage{braket}
\usepackage{amssymb}% http://ctan.org/pkg/amssymb
\usepackage{pifont}% http://ctan.org/pkg/pifont
\newcommand{\cmark}{\ding{51}}%
\newcommand{\xmark}{\ding{55}}%

\setlength{\parindent}{0em} 

\title{Theoretische Grundlagen der Informatik\\ Zusammenfassung}
\author{Jean-Pierre von der Heydt}
\date{\today}

\def \dozent{Prof. Dr. Dorothea Wagner}
\def \texcol{RoyalBlue}

\newcommand{\tcol}[1]{\textcolor{RoyalBlue}{#1}}
\newcommand{\set}[1]{\left\lbrace #1\right\rbrace}

\theoremstyle{break}
\theorembodyfont{\upshape}
\newtheorem{satz}{Satz}[section]
\newtheorem{defi}[satz]{Definition}
\newtheorem{beis}[satz]{Beispiel}
\newtheorem{verf}[satz]{Verfahren}
\newtheorem{prob}[satz]{Problem}


\begin{document}

\maketitle
\pagebreak

\tableofcontents
\pagebreak



\section{Einführung} 

Diese Zusammenfassung beruht auf der Vorlesung Theoretische Grundlagen der Informatik in WS 17/18 und dem dazugehörigen Skript von Prof. Dr. Dorothea Wagner.

\subsection{Alphabete\index{Alphabet} und Wörter\index{Wort}}

\begin{defi}[Definitionen zu Alphabeten und Wörtern]
\begin{itemize}
\item Ein Alphabet $\Sigma$ ist eine endliche Menge von Zeichen.
\item Ein Wort w über $\Sigma$ ist eine (möglicherweise leere) Folge von Zeichen aus $\Sigma$.
\item Das leere Wort\index{leere Wort} wird mit $\epsilon$ symbolisiert.
\item Die Menge aller Wörter über $\Sigma$ wird mit $\Sigma^*$ abgekürzt.
\item Die Menge aller Wörter der Länge $n$ über $\Sigma$ wird mit $\Sigma^n$ abgekürzt.
\item Die Konkatenation\index{Konkatenation} zweier Wörter $w_1$ und $w_2$ wird mit $w_1\cdot w_2$ oder $w_1w_2$ abgekürzt.
\item Die iterierte Konkatenation eines Wortes $w$ bezweichnen wir mit $w^k\coloneqq \underbrace{w\cdot\dots\cdot w}_{k\text{ mal}}$
\item Lässt sich $w$ als $w=t\cdot u\cdot v$, so heißt:
$\left. \begin{array}{cc} 	
							t &\text{ Präfix}\\
							u &\text{ Teilwort}\\
							v &\text{ Suffix} 
\end{array} \right\}$
von $w$
\end{itemize}
\end{defi}


\subsection{Formale Sprachen\index{formale Sprachen}}

\begin{defi}[Definitionen zu formalen Sprachen]
\begin{itemize}
\item Eine formale Sprache $L$ über $\Sigma$ ist eine Teilmenge $L\subseteq\Sigma^*$
\item Die Produktsprache\index{Produktsprache} zweier Sprachen $L_1$ und $L_2$ definieren wir als\\ $L_1\cdot L_2\coloneqq\set{w_1w_2\mid w_1\in L_1\text{ und } w_2\in L_2}$
\item Das Produkt von $L$ mit sich selbst wird als $L^k\coloneqq \underbrace{L\cdot\dots\cdot L}_{k\text{ mal}}$ bezeichnet. $L^0 = \set{\epsilon}$
\item Der Kleene'sche Abschluss\index{Kleene'scher Abschluss} ist definiert durch $L^*\coloneqq \bigcup\limits_{i\geq 0} L^i$ 
\item Der positive Kleenne'sche Abschluss ist definiert durch $L^+\coloneqq \bigcup\limits_{i\geq 1} L^i$ 
\item Die Quotientensprache\index{Quotientensprache} bezeichnet $L_1/L_2\coloneqq\set{w\in \Sigma^*\mid\exists z\in L_2\text{ mit } w\cdot z\in L_1}$
\item Die Komplementsprachen\index{Komplementsprache} bezeichnet $L^c\coloneqq\Sigma^*\setminus L$
\end{itemize}
\end{defi}


\subsection{Reguläre Sprachen\index{reguläre Sprache}}

\begin{defi}[Reguläre Sprache] 
Eine Sprache $L$ über $\Sigma$ heißt regulär, wenn sie auf folgende Weise produziert werden kann:
\begin{itemize}
\item Verankerung:
\begin{itemize}
\item $L=\set{a}\text{ mit } a\in\Sigma$
\item $L=\set{}$
\item $L=\set{\epsilon}$
\end{itemize}
\item Induktion: $L_1,L_2$ seien regulär
\begin{itemize}
\item $L=L_1\cdot L_2$
\item $L=L_1\cup L_2$
\item $L=L_1^*$
\end{itemize}
\end{itemize}
\end{defi}


\subsection{Kontextfreie Sprachen\index{kontextfreie Sprache}}

\begin{defi}[Kontextfreie Grammatik]
Eine kontextfreie Grammatik\index{kontextfreie Grammatik} $G=(\Sigma ,V,S,R)$ ist gegeben durch:
\begin{itemize}
\item ein endliches Alphabet $\Sigma$
\item eine endliche Menge $V$ mit $V\cap\Sigma = \set{}$ von \tcol{Nichtterminalsymbolen}\index{Nichtterminalsymbol}
\item ein \tcol{Startsymbol}\index{Startsymbol} $S\in V$
\item eine endliche menge von \tcol{Ableitungsregeln}\index{Ableitungsregel} $R$, d.h. eine Menge von Tupeln $(I,r)\in V\times (\Sigma\cup V)^*$
\end{itemize}
Wir schreiben auch $I\rightarrow r$. $L(G)\coloneqq \set{w\in\Sigma^*\mid S\rightarrow^*w}$ bezeichnet die von $G$ erzeugte Sprache.
\end{defi}



\section{Endliche Automaten\index{endlicher Automat} und reguläre Ausdrücke\index{regulärer Ausdruck}}


\subsection{Deterministische endliche Automaten\index{deterministischer endlicher Automat}}

\begin{defi}[DEA]
Ein (deterministischer) endlicher Automat $(Q,\Sigma ,\delta ,s,F)$ (D)EA besteht aus:
\begin{itemize}
\item $Q$, einer endlichen Menge von \tcol{Zuständen}\index{Zustand}
\item $\Sigma$, einer endlichen Menge von \tcol{Eingabesymbolen}\index{Eingabealphabet}
\item $\delta\colon Q\times\Sigma\to Q$, einer \tcol{Übergangsfunktion}
\item $s\in Q$, eines \tcol{Startzustand}\index{Startzustand}
\item $F\subseteq Q$, einer Menge von \tcol{Endzuständen}\index{Endzustand}
\end{itemize}
\end{defi}

\begin{beis}
In diesem Beispiel ist $s=q_a$ und $F=\set{q_d}$. Das Eingabealphabet ist $\set{0,1}$:\\
\begin{center}
\begin{tikzpicture}[->,>=stealth',auto,node distance=2.8cm,semithick]

  \node[initial,state] 	(A)                    {$q_a$};
  \node[state]         	(B) [above right of=A] {$q_b$};
  \node[state,accepting](D) [below right of=A] {$q_d$};
  \node[state]         	(C) [below right of=B] {$q_c$};

  \path (A) edge              node {0} (B)
            edge              node {1} (C)
        (B) edge [loop above] node {1} (B)
            edge              node {0} (C)
        (C) edge              node {0,1} (D)
        (D) edge [loop below] node {1} (D)
            edge              node {0} (A);
\end{tikzpicture}
\end{center}
\end{beis}

\begin{defi}
Ein EA \tcol{erkennt} oder \tcol{akzeptiert} eine Sprache $L$ (über dem Alphabet des Automaten), wenn er nach Abarbeitung eines Wortes $w$ genau dann in einem Endzustand ist, wenn $w$ in $L$ ist.
\end{defi}

\begin{satz}
Jede reguläre Sprache\index{reguläre Sprache} wird von einem (deterministischen) endlichen Automaten\index{endlicher Automat} akzeptiert.
\end{satz}


\subsection{Nichtdeterministische endliche Automaten}

\begin{defi}[NEA]
Ein nichtdeterministischer endlicher Automat $(Q,\Sigma ,\delta ,s,F)$ NEA besteht aus:
\begin{itemize}
\item $Q$, einer endlichen Menge von \tcol{Zuständen}\index{Zustand}
\item $\Sigma$, einer endlichen Menge von \tcol{Eingabesymbolen}\index{Eingabealphabet}
\item $\delta\colon Q\times (\Sigma\cup\set{\epsilon})\to 2^Q$, einer \tcol{Übergangsfunktion}, wobei $2^Q$ die Potenzmenge\index{Potenzmenge} von $Q$ darstellt
\item $s\in Q$, eines \tcol{Startzustand}\index{Startzustand}
\item $F\subseteq Q$, einer Menge von \tcol{Endzuständen}\index{Endzustand}
\end{itemize}
\end{defi}

\begin{defi} 
Ein NEA \tcol{erkennt} oder \tcol{akzeptiert} eine Sprache $L$ (über dem Alphabet des Automaten), wenn es genau dann eine Folge von Übergängen für ein Wort $w$ gibt, die $w$ akzeptiert, wenn $w$ in $L$ ist.
\end{defi}

\begin{defi} 
Zwei endliche Automaten heißen \tcol{äquivalent}\index{äquivalente Automaten}, wenn sie dieselbe Sprache akzeptieren.
\end{defi}

\begin{satz}[Äquivalenz von NEA's und DEA's]
Zu jedem nichtdeterministischen endlichen Automaten gibt es einen äquivalenten deterministischen Automaten.
\end{satz}

\begin{verf}[Potenzmengenkonstruktion\index{Potenzmengenkonstruktion}]
Wir konstruieren für jeden NEA $A\coloneqq (Q,\Sigma ,\delta ,s,F)$ Einen DEA $\tilde{A}\coloneqq (\tilde{Q},\Sigma ,\tilde{\delta},\tilde{s},\tilde{F})$ auf folgende Weise:
\begin{itemize}
\item definieren für jeden Zustand $q\in Q$ den \tcol{$\epsilon$-Abschluss}\index{$\epsilon$-Abschluss} \[E(q)\coloneqq \set{p\in Q\mid p \text{ ist von } q \text{ durch } \epsilon\text{-Übergänge erreichbar}}\]
\item $\tilde{Q} = 2^Q$
\item $\tilde{\delta}\colon\tilde{Q}\times\Sigma\to\tilde{Q}$ wobei $\tilde{\delta}(\tilde{q},a)$ die Menge an Zuständen ist, die der NEA, startend von einem Zustand aus $\tilde{q}$ aus nach abarbeiten von beliebig vielen $\epsilon$-Übergängen \tcol{und} dem Symbol $a$ erreichen kann.
\item $\tilde{s}\coloneqq E(s)$
\item $\tilde{F}\coloneqq \set{\tilde{q}\in\tilde{Q}\mid\tilde{q}\cap F\neq\set{}}$
\end{itemize}
\end{verf}

\begin{satz}[NEA \tcol{ohne $\epsilon$-Übergänge}\index{$\epsilon$-Übergang}]
Zu jedem NEA mit $\epsilon$-Übergängen gibt es einen äquivalenten NEA ohne $\epsilon$-Übergänge, der nicht mehr Zustände hat.
\end{satz}

\begin{verf}[Konstruktion für NEA ohne $\epsilon$-Übergänge]
%TODO Sind die Zustandsmengen nicht einfach immer Gleich? Also \tilde{Q}=Q?
Für einen NEA $A\coloneqq (Q,\Sigma ,\delta ,s,F)$ konstruieren wir einen NEA ohne $\epsilon$-Übergänge $\tilde{A}\coloneqq (\tilde{Q},\Sigma ,\tilde{\delta},\tilde{s},\tilde{F})$ auf folgende Weise:
\begin{itemize}
\item $\tilde{Q}\coloneqq (Q\setminus F)\cup\tilde{F}$
\item $\tilde{s}\coloneqq s$
\item $\tilde{\delta}(q,a)$ soll genau die Zustände enthalten, die $A$ startend von $q$ aus nach abarbeiten von beliebig vielen $\epsilon$-Übergängen \tcol{und} dem Symbol $a$ erreichen kann.
\item $\tilde{F}\coloneqq\set{q\mid E(q)\cap F\neq\set{}}$
\end{itemize}
\end{verf}

\begin{satz}
Jede Sprache, die von einem endlichen Automaten\index{endlicher Automat} erkannt wird, ist regulär\index{reguläre Sprache}.
\end{satz}

\begin{verf}[Konstruktion eines regulären Ausdrucks\index{regulärer Ausdruck} aus einem DEA\index{deterministischer endlicher Automat}]
Wir betrachten folgende Menge:
\[L_{q_r,i,q_t}\coloneqq\set{w\in\Sigma^*\mid\text{Abarbeitung von }w\text{ führt von }q_r\text{ nach }q_t\text{ und hat nur Zwischenzustände }\set{q_1,\dots ,q_i}}\]
Nun betrachten wir:
\[L_{q_r,0,q_t}=
	\begin{cases}
		\set{a\in\Sigma\mid\delta (q_t,a)=q_t}\cup\set{\epsilon} & \text{falls } r=t\\
		\set{a\in\Sigma\mid\delta (q_r,a)=q_t} & \text{sonst}
	\end{cases}\]
Für $i>0$ gilt dann:
\[L_{q_r,i+1,q_t}=L_{q_r,i,q_t}\cup (L_{q_r,i,q_{i+1}}\cdot (L_{q_{i+1},i,q_{i+1}})^*\cdot L_{q_{i+1},i,q_t})\]
Damit ist dann auch $L_f=L_{s,n,f}$ regulär für jedes $f\in F$.
\end{verf}

\begin{satz}[Pumping-Lemma für reguläre Sprachen\index{Pumping-Lemma für reguläre Sprachen}]
Sei $L$ eine reguläre Sprache. Dann existiert eine Zahl $n\in\mathbb{N}$, so dass für jedes Wort $w\in L$ mit $|w|>n$ eine Darstellung
\[w=uvx \text{ mit } |uv|\leq n, v\neq\epsilon,\]
existiert, bei der auch $uv^ix\in L$ ist für alle $i\in\mathbb{N}_0$.
\end{satz}

\begin{satz}[Verallgemeinertes Pumping-Lemma für reguläre Sprachen\index{Pumping-Lemma für reguläre Sprachen}]
Sei $L$ eine reguläre Sprache. Dann existiert eine Zahl $n\in\mathbb{N}$, so dass für jedes Wort $w\in L$ mit $|w|>n$ und jede Darstellung $w=tyx$ mit $|y|=n$ gilt:\\
Für das Teilwort $y$ existiert eine Darstellug $y=uvz$ mit $v\neq\epsilon$ bei der auch $tuvîzx\in L$ ist für alle $i\in\mathbb{N}_0$.
\end{satz}


\subsection{Minimierung von Automaten\index{Minimierung von Automaten} und Äquivalenzklassenautomaten\index{Äquivalenzklassenautomaten}}

\begin{defi}[Überflüssige Zustände\index{überflüssige Zustände}]
Zustände eines (deterministischen) endlichen Automaten\index{endlicher Automat}, die vom Anfangszustand aus nicht erreichbar sind, heißen \tcol{überflüssig}
\end{defi}

\begin{satz}
Die Menge aller \tcol{überflüssigen Zustände} eines (deterministischen) endlichen Automaten\index{endlicher Automat} kann in Zeit $O(|Q|\cdot |\Sigma)$ berechnet werden (\tcol{DFS}).
\end{satz}

\begin{defi}[Äquivalente Zustände\index{äquivalente Zustände}]
Zwei Zustände $p$ und $q$ eines deterministischen endlichen Automaten heißen \tcol{äquivalent}$(p\equiv q)$, wenn für alle Wörter $w\in\Sigma^*$ gilt:
\[\delta(p,w)\in F\Leftrightarrow\delta(q,w)\in F\]
Mit $[q]$ bezeichnen wir die \tcol{Äquivalenzklasse} der zu $q$ äquivalenten Zustände.
\end{defi}

\begin{verf}[Konstruktion des \tcol{Äquivalenzklasseautomaten}\index{Äquivalenzklasseautomat}]
Zu einem DEA $A=(Q,\Sigma,\delta ,s,F)$ definieren wir den Äquivalenzklasseautomaten $A^\equiv=(Q^\equiv ,\Sigma^\equiv ,\delta^\equiv ,s^\equiv ,F^\equiv)$ durch:
\begin{itemize}
\item $Q^\equiv\coloneqq\set{[q]\mid q\in Q}$
\item $\Sigma^\equiv\coloneqq\Sigma$
\item $\delta^\equiv([q],a)\coloneqq[\delta(q,a)]$
\item $s^\equiv\coloneqq[s]$
\item $F^\equiv\coloneqq\set{[f]\mid f\in F}$
\end{itemize}
Um die Äquivalenzklassen zu bestimmen unterteile $Q$ in immer kleinere Klassen von Zuständen, indem zuerst ein \tcol{Zeuge}\index{Zeuge} für die nichtäquivalenz zweier Zustände der Länge $0$, dann $1$, dann $2\dots$ gesucht wird. Das Verfahren bricht ab, wenn für Wörter der Länge $k$ keine Zeugen mehr gefunden werden.\\
Die enstandenen Klassen sind die gesuchten \tcol{Äquivalenzklassen}.
\end{verf}

\begin{satz}
Der Äquivalenzklasseautomat\index{Äquivalenzklasseautomat} $A^\equiv$ zu $A$ akzeptiert dieselbe Sprache wie $A$.
\end{satz}

\begin{satz}
Der Äquivalenzklasseautomat\index{Äquivalenzklasseautomat} $A^\equiv$ zu $A$ ohne überflüssige Zustände ist \tcol{minimal}.
\end{satz}

\begin{defi}[Rechtsinvariante Äquivalenzrelation\index{rechtsinvariante Äquivalenzrelation}]
Eine Äquivalenzrelation $R$ über $\Sigma$ heißt \tcol{rechtsinvariant}, wenn für alle $x,y\in\Sigma^*$ gilt:
\[\text{falls } x\ R\ y\text{ so gilt auch }xz\ R\ yz\text{ für alle } z\in\Sigma^*\]
Der \tcol{Index} von $R$ $ind(R)$ ist die Anzahl der Äquivalenzklassen von $\Sigma^*$ bezüglich $R$.
\end{defi}

\begin{defi}[Nerode-Relation\index{Nerode-Relation}]
Für eine Sprache $L\subseteq\Sigma^*$ ist die \tcol{Nerode-Relation} $R_L$ definiert durch:\\
für $x,y\in\Sigma^*$ ist $x\ R\ y$ genau dann, wenn $(xz\in L\Leftrightarrow yz\in L)$ für alle $z\in\Sigma^*$ gilt.\\
Die Nerode-Relation ist offensichtlich \tcol{rechtsinvariant}.
\end{defi} 

\begin{satz}[von Nerode]
Die folgenden Aussagen sind äquivalent:
\begin{itemize}
\item $L\subseteq\Sigma^*$ wird von einem deterministischen endlichen Automaten erkannt.
\item $L$ ist die Vereinigung von (einigen) Äquivalenzklassen einer rechtsinvarianten Äquivalenzrelation mit endlichem Index.
\item Die Nerode-Relation hat endlichen Index.
\end{itemize}
\end{satz}



\section{Turingmaschine\index{Turingmaschine} und Berechenbarkeit}


\subsection{Die Turingmaschine\index{Turingmaschine}}

\begin{defi}[Turingmaschine\index{Turingmaschine}]
Eine deterministische \tcol{Turingmaschine} ((D)TM) besteht aus:
\begin{itemize}
\item $Q$, einer endlichen \tcol{Zustandsmenge}
\item $\Sigma\cup\set{\sqcup}$, einem \tcol{endlichen Eingabealphabet} und \tcol{Blanksymbol}
\item $\Gamma$, einem endlichen \tcol{Bandalphabet} mit $\Sigma\cup\set{\sqcup}\subseteq\Gamma$
\item $s\in Q$, einem \tcol{Startzustand}
\item $\delta\colon Q\times\Gamma\to Q\times \Gamma\times\set{L,R,N}$, einer \tcol{Übergangsfunktion}. Hierbei stehen $L,R,N$ für eine Bewegung des Kopfes nach Links, Rechts oder ein Stehenbeleiben
\item $F\subseteq Q$, einer Menge von \tcol{Endzuständen}. Diese Menge kann auch entfallen.\\
Beachte: Die TM stopp, sobald sie in einen Endzustand gerät
\end{itemize}
Die TM stopp auch, wenn sie im Zustand $q$ ein $a$ ließt und $\delta(q,a)=(q,a,N)$ ist.
\end{defi}

\begin{beis}
Der Übergang $\delta(q,a)=(p,b,L)$ wird graphisch wie folgt dargestellt:
\begin{center}
\begin{tikzpicture}[->,>=stealth',auto,node distance=2.8cm,semithick]

  \node[state] 	(A)  {$q$};
  \node[state]  (B) [right of=A]  {$p$};

  \path (A) edge node {$a|b,\, L$} (B);

\end{tikzpicture}
\end{center}
Wenn für eine bestimmte Kombination $q,a$ kein Übergang angegeben ist, stopp die TM in einem nicht akzeptierenden Zustand
\end{beis}

\begin{defi}[Definitionen zur Turingmaschine\index{Turingmaschine}]
\begin{itemize}
\item Eine TM \tcol{akzeptiert}\index{Turingmaschine akzeptiert} eine Eingabe $w\in\Sigma^*$, wenn sie nach Lesen von $w$ in einem Zustand aus $F$ stoppt
\item Eine TM \tcol{akzeptiert} eine Sprache $L$ genau dann, wenn sie ausschließlich Wörter $w\in L$ als Eingabe akzeptiert.
\item Eine Sprache $L\subseteq\Sigma^*$ heißt \tcol{rekursiv}\index{rekursiv} oder \tcol{entscheidbar}\index{entscheidbar}, wenn es eine TM gibt, die auf allen Eingaben stopp und eine Eingabe $w$ genau dann akzeptiert, wenn $w\in L$ gilt
\item Eine Sprache $L\subseteq\Sigma^*$ heißt \tcol{rekursiv-aufzählbar}\index{rekursiv-aufzählbar} oder \tcol{semi-entscheidbar}\index{semi-entscheidbar}, wenn es eine TM gibt, die genau die Eingaben von $w$ akzeptiert für die $w\in L$.\\
Das Verhalten der TM für Eingaben $w\notin L$ ist damit nicht definiert.
\item Eine Funktion $f\colon\Sigma^*\to\Gamma^*$ heißt \tcol{berechenbar}\index{berechenbar} oder \tcol{totalrekursiv}\index{totalrekursiv}, wenn es eine TM gibt, die bei Eingabe von $w\in\Sigma^*$ den Funktionswert $f(w)\in\Gamma^*$ ausgibt.
\item Eine TM \tcol{realisiert} die Funktion $f\colon\Sigma^*\to\Gamma^*$, mit $f(w)=$ Ausgabe der TM, wenn sie stoppt und undefiniert sonst.
\end{itemize}
\end{defi}

Die \tcol{Chruch'sche These}\index{Chruch'sche These} besagt, dass die Menge der berechenbaren Funktionen genau die Menge der im inuitiven Sinne überhaupt berechenbaren Funktionen ist.\\

Es gibt verschiedene \tcol{Varianten} der TM, die alle gleichmächtig zur TM sind:
\begin{itemize}
\item Mehrere Lese-/Schreibköpfe
\item Mehrere Bänder
\item Mehrere Lese-/Schreibköpfe für mehrere Bänder
\item Mehrdimensionale Bänder
\end{itemize}


\subsection{Die universelle Turingmaschine\index{universelle Turingmaschine} und unentscheidbare Probleme\index{Unentscheidbarkeit}}

\begin{defi}[Gödelnummer\index{Gödelnummer}]
Sei $M=(Q,\Sigma,\Gamma,\delta,s,F)$ eine Turingmaschine. Die \tcol{Gödelnummer} von $M$ wird mit $\langle M\rangle$ bezeichnet. Es gilt $\langle M\rangle\in\set{0,1}^*$ und $\langle M\rangle$ definiert $M$ vollständig.\\
$T_w$ sei die TM, mit der Gödelnummer $w$, beziehungsweise die TM, die $\set{}$ akzeptiert.
\end{defi}

\begin{defi}[universelle Turingmaschine\index{universelle Turingmaschine}]
Eine universelle Turingmaschine erhält als Eingabe ein Paar $(\langle M\rangle,w)$ und Simuliert $M$ auf $w$.
\end{defi}

\begin{satz}[Unentscheidbarkeit\index{Unentscheidbarkeit} der Diagonalsprache\index{Diagonalsprache}]
	Die Diagonalsprache ist definiert durch:
	\[L_d\coloneqq\set{w_i\mid T_{w_i}\text{ akzeptiert }w_i\text{ nicht}}\]
	Die Diagonalsprache $L_d$ ist nicht semi-entscheidbar.\\
	$L_d^c$ ist semi-entscheidbar\index{semi-entscheidbar}.
\end{satz}

\begin{satz}[Unentscheidbarkeit\index{Unentscheidbarkeit} des Halteproblem\index{Halteproblem}]
Das \tcol{Halteproblem} definiert folgende Sprache
\[\mathcal{H}\coloneqq\set{wv\mid T_w\text{ hält auf der Eingabe }v}\]
und ist semi-entscheidbar.\\
$H^c$ ist nicht semi-entscheidbar\index{semi-entscheidbar}.
\end{satz}

\begin{satz}[Unentscheidbarkeit\index{Unentscheidbarkeit} der universellen Sprache\index{universelle Sprache}]
Die \tcol{universelle Sprache} ist definiert durch
\[L_u\coloneqq\set{wv\mid v\in L(T_w)}\]
und ist nicht entscheidbar aber \tcol{semi-entscheidbar}\index{semi-entscheidbar}.
\end{satz}

\begin{satz}[Satz von Rice\index{Satz von Rice}]
Sei $R$ die Menge der von Turingmaschinen berechenbaren Funktionen und $S$ eine nichttriviale Teilmenge von $R$ ($\set{}\subsetneq S\subsetneq R$). Dann ist die Sprache
\[L(S)\coloneqq\set{\langle M\rangle\mid M\text{ berechnet eine Funktion aus }S}\]
nicht entscheidbar.
\end{satz}

\begin{satz}[Eigenschaften von (semi-)entscheidbaren Sprachen\index{entscheidbar}\index{semi-entscheidbar}]
\begin{itemize}
\item Die entscheidbaren Sprachen sind abgeschlossen unter Komplementbildung, Schnitt, Vereinigung und Mengendifferenz
\item Die semi-entscheidbaren Sprachen sind abgeschlossen unter Schnitt und Vereinigung, aber \tcol{nicht unter Komplementbildung oder Mengendifferenz}
\item $L$ und $L^c$ semi-entscheidbar genau dann, wenn $L$ entscheidbar
\end{itemize}
\end{satz}

\begin{prob}[Post'sche Korrespondenzproblem\index{Post'sche Korrespondenzproblem}]
Gegeben ist eine endliche Folge von Wortpaaren
\[K=((x_1,y_1),\dots,(x_n,y_n))\]
über einem endlichen Aplhabet $\Sigma$. Es gilt $x_i\neq\epsilon$ und $y_i\neq\epsilon$. Gefragt ist, ob eine endliche Folge von Indizes $i_1,\dots,i_k\in\set{1,\dots,n}$ existiert, so dass $x_{i_1}\dots x_{i_k}=y_{i_1}\dots y_{i_k}$ gilt.\\
Dieses Problem ist nicht entscheidbar.
\end{prob}



\section{Komplexitätsklassen\index{Komplexitätsklassen}}


\subsection{Zeitkomplexität}

\begin{defi}[Zeitkomplexitätsfunktion eine deterministischen Turingmaschine\index{Zeitkomplexitätsfunktion einer TM}]
Für eine deterministische Turingmaschine $M$, die für alle Eingaben über $\Sigma$ hält, ist die \tcol{Zeitkomplexitätsfunktion} $T_M\colon\mathbb{Z}^+\to\mathbb{Z}^+$ definiert durch:
\[
T_M(n)=\max\set{m\,\middle|\,
\begin{array}{c}
\text{es gibt eine Eingabe $x\in\Sigma^*$ mit $|x|=n$, so dass $M$ $m$}\\
\text{Berechnungsschritte benötigt, um in einen Endzustand zu gelangen}
\end{array}}
\]
\end{defi}

\begin{defi}[Die Klasse P\index{Problemklasse P}]
Die Klasse $P$ ist die Menge aller Sprachen, für $L$ (Probleme) für die es eine deterministische TM gibt, deren Zeitkomplexitätsfunktion polynomial ist. Das heißt es existiert ein Polynom $p$ mit
\[T_M(n)\leq p(n).\]
\end{defi}


\subsection{Die Nichtdeterministische Turingmaschine\index{nichtdeterministische Turingmaschine} und die Klasse NP\index{Problemklasse NP}}

\begin{defi}[Nichtdeterministische Turingmaschine\index{nichtdeterministische Turingmaschine}]
Die \tcol{nichtdeterministische Turingmaschine} (NTM) wird analog zur TM definiert hat aber zusätzlich zu der endlichen Kontrolle mit Lese-/Schreibkopf ein \tcol{Orakelmodul} mit eigenem Schreibkopf.\\
Bei der Berechnung der DTM schreibt das Orakelmodul zuerst ein beliebiges Wort aus $\Gamma^*$ auf das Band und gibt anschließend die Kontrolle an den deterministischen Teil über.
\end{defi}

\begin{defi}
Eine NTM akzeptiert ein Wort $w\in\Sigma^*$ genau dann, wenn es eine Berechnung gibt, die in einem akzeptierenden Zustand endet.\\
Sie akzeptiert eine Sprachen $L\subseteq\Sigma^*$ genau dann, wenn sie grade die Wörter aus $L$ akzeptiert.
\end{defi}

\begin{defi}[Zeitkomplexitätsfunktion eine nichtdeterministischen Turingmaschine\index{Zeitkomplexitätsfunktion einer NTM}]
Die Zeitkomplexitätsfunktion einer nichtdeterministische Turingmaschine $M$ $T_M\colon\mathbb{Z}^+\to\mathbb{Z}^+$ ist definiert durch:
\[
T_M(n)=\max\left(\set{1}\cup\set{m\,\middle|\,
\begin{array}{c}
\text{es gibt eine Eingabe $x\in\Sigma^*$ mit $|x|=n$, so dass $M$ $m$}\\
\text{Berechnungsschritte benötigt, um $x$ zu akzeptieren}
\end{array}}\right)
\]
Zur Berechung von $T_M(n)$ wird für jedes $x\in L_M$ mit $|x|=n$ die kürzeste akzeptierende Berechnung betrachtet und unter diesen $x$ dann die längste Berechnung bestimmt. Die Schreibzeit des Orakelmodul spielt mit in die gesamte Berechnungszeit ein.\\
Die Zeitkomplexität hängt nur von der Anzahl der Schritte bei einer akzeptierenden Berechnung ab.
\textbf{Bemerkung vom Autor:} das ist ziemlich obskur definiert.
\end{defi}

\begin{defi}[Die Klasse NP\index{Problemklasse NP}]
Die Klasse $NP$ ist die Menge aller Sprachen (Probleme) $L$, für die es eine nichtdeterministische Turingmaschine gibt, deren Zeitkomplexitätsfunktion polynomial beschränkt ist.\\
Alle Sprachen in $NP$ sind entscheidbar.
\end{defi}

\begin{defi}[polynomiale Transformation\index{polynomiale Transformation}]
Eine \tcol{polynomiale Transformation} einer Sprache $L_1\subseteq\Sigma_1^*$ in eine Sprache $L_2\subseteq\Sigma_2^*$ ist eine Funktion $f\colon\Sigma_1^*\to\Sigma_2^*$ mit den Eigenschaften:
\begin{itemize}
\item es existiert eine polynomiale TM, die $f$ berechnet
\item für alle $x\in\Sigma_1^*$ gilt: $x\in L_1\Leftrightarrow f(x)\in L_2$
\end{itemize}
Wir schreiben dann $L_1\propto L_2$.
\end{defi}

\begin{defi}[$NP$-vollständig\index{NP-vollständig}]
Eine Sprache $L$ heißt $NP$-vollständig, wenn $L\in NP$ und für alle $L'\in NP$ gilt $L'\propto L$.
\end{defi}

\begin{satz}
Sei $L$ $NP$-vollständig, dann gilt:
\begin{itemize}
\item $L\in P\Rightarrow P=NP$
\item $L\notin P$, so gillt für alle $NP$-vollständigen Sprachen $L'$, dass $L'\notin P$.
\end{itemize}
\end{satz}

%TODO Orakelturingmaschine und Turing-Reduzierbarkeit 

\begin{defi}[$NP$-schwer\index{NP-schwer}]
Ein Suchproblem $\Pi$ heißt $NP$-schwer, falle es eine $NP$-vollständige Sprache $L$ gibt, mit $L\propto_T\Pi$. Also kann mit einem Algorithmus für $\Pi$ auch $L$ gelöst werden.
\end{defi}


\subsection{Komplementsprachen}

\begin{defi}
Die Klasse \tcol{$NPC$}\index{NPC} sei die Klasse der $NP$-vollständigen Sprachen/Probleme.\\
Die Klasse \tcol{$NPI$}\index{NPI} ist definiert durch $NPI\coloneqq NP\setminus(P\cup NPC)$.\\
Die Klasse \tcol{$co-P$}\index{co-P} ist die Klasse aller Sprachen $\Sigma^*\setminus L$ für $L\subseteq\Sigma^*$ und $L\in P$ (die Klasse der Komplementsprachen).\\
Die Klasse \tcol{$co-NP$}\index{co-NP} ist die Klasse aller Sprachen $\Sigma^*\setminus L$ für $L\subseteq\Sigma^*$ und $L\in NP$
\end{defi}


\subsection{Pseudopolynomielle Algorithmen\index{pseudopolynomielle Algorithmen}}

\begin{defi}[Pseudopolynomiell\index{pseudopolynomielle Algorithmen}]
Sei $\Pi$ ein Optimierungsproblem. Ein Algorithmus, der $\Pi$ löst, heißt pseudopolynomiell, falls seine Laufzeit durch ein Polynom der \tcol{Eingabegröße} und der \tcol{Größe der größten in der Eingabe vorkommenden Zahl} beschränkt ist.
\end{defi}

\begin{defi}[stark $NP$-vollständig\index{stark NP-vollständig}]
Ein Entscheidungsproblem\index{Entscheidungsproblem} $\Pi$ heißt stark $NP$-vollständig, wenn das Teilproblem $\Pi_p$, in dem nur Instanzen vorkommen, bei denen die größte Zahl polynomiell durch die Eingabelänge beschränkt ist, $NP$-vollständig ist.
\end{defi}

\begin{satz}
Ist $\Pi$ stark $NP$-vollständig und $NP\neq P$, dann gibt es keinen pseudopolynomiellen Algorithmus für $\Pi$.
\end{satz}


\subsection{Approximationsalgorithmen\index{Approximationsalgorithmen} für Optimierungsprobleme}

\begin{defi}
Für $I\in D_\Pi$ bezeichne $OPT(I)$ den Wert der Optimallösung. Zu einem Algorithmus $A$ bezeichne $A(I)$ den Wert, den $A$ für $I$ ausgibt.
\end{defi}

\begin{defi}[Absoluter Approximationsalgorithmus\index{absoluter Approximationsalgorithmus}]
Sei $\Pi$ ein Optimierungsproblem. Ein polynomialer Algorithmus $A$, der für jedes $I\in D_\Pi$ einen Wert $A(I)$ liefert, mit
\[|OPT(I)-A(I)|\leq K\]
für $K\in\mathbb{N}_0$ konstant, heißt absoluter Approximationsalgorithmus.\\
Es gibt nur wenige $NP$-schwere Probleme, für die diese Art von Algorithmen bekannt sind.
\end{defi}

\begin{defi}[Approximationsalgorithmus mit relativer Gütegarantie\index{relativer Approximationsalgorithmus}]
Sei $\Pi$ ein Optimierungsproblem. Ein polynomialer Algorithmus $A$, der für jedes $I\in D_\Pi$ einen Wert $A(I)$ liefert, mit $R_A(I)\leq K$, wobei $K\geq 1$ eine Konstante, und
\[R_A(I)\coloneqq	
	\begin{cases}
		\frac{A(I)}{OPT(I)} & \text{falls $\Pi$ Minimierungsproblem}\\
		& \\
		\frac{OPT(I)}{A(I)} & \text{falls $\Pi$ Maximierungsproblem}
	\end{cases}
\]
heißt Approximationsalgorithmus mit relativer Gütegarantie.
\end{defi}

\begin{defi}
Zu einem polynomialen Approximationysalgorithmus $A$ sei
\[R^{\,\infty}_A\coloneqq\set{r\geq 1 \mid\text{es gibt ein $N_0>0$, so dass $R_A(I)\leq r$ für alle $I$ mit $OPT(I)\geq N_0$}}\]
\end{defi}

\begin{defi}[Approximationsschemata\index{Approximationsschema}]
Ein (polynomiales) Approximationsschemata (PAS) für ein Optimierungsproblem $\Pi$ ist eine \tcol{Familie von Algorithmen} $\set{A_\epsilon\mid\epsilon>0}$, so dass für alle $\epsilon>0$
\begin{itemize}
\item $R_{A_\epsilon}\leq 1+\epsilon$ ist (d.h. $A_\epsilon$ ist ein $\epsilon$-approximierender Algorithmus).
\item $A_\epsilon$ polynomial in der Größe des Inputs ist.
\end{itemize}
Ein Approximationsschema $\set{A_\epsilon\mid\epsilon>0}$ heißt \tcol{vollpolynomial}\index{vollpolynomiales Approximationsschema} (FPAS) falls seine Laufzeit zudem polynomial in $\frac{1}{\epsilon}$ ist.
\end{defi}

\begin{satz}
Sei $\Pi$ ein $NP$-schweres Optimierungsproblem mit:
\begin{itemize}
\item $OPT(I)\in\mathbb{N}$ für alle $I\in D_\Pi$
\item es existiert ein Polynom $q$ mit $OPT(I)<q(\langle I\rangle)$ für alle $I\in D_\Pi$ wobei $\langle I\rangle$ die Inputlänge von $I$ ist.
\end{itemize}
Falls $P\neq NP$, so gibt es kein FPAS $\set{A_\epsilon\mid\epsilon>0}$ für $\Pi$.
\end{satz}

\begin{satz}
Sei $\Pi$ ein Optimierungsproblem und $\max(I)$ die größte in $\Pi$ vorkommende Zahl. Wenn gilt:
\begin{itemize}
\item $OPT(I)\in\mathbb{N}$ für alle $I\in D_\Pi$
\item es existiert ein Polynom $q$ mit $OPT(I)\leq q(\langle I\rangle,\max(I))$
\end{itemize}
Dann hat $\Pi$ genau dann ein FPAS, wenn es zu $\Pi$ einen \tcol{pseudopolynomialen optimalen Algorithmus}\index{pseudopolinomielle Algorithmen} gibt
\end{satz}



\section{Grammatiken und die Chomsky-Hierarchie\index{Chomsky-Hierarchie}}


\begin{defi}[Grammatik\index{Grammatik}]
Eine Grammatik $G=(\Sigma,V,S,R)$ besteht auf 4 Komponenten:
\begin{itemize}
\item Endliches \tcol{Alphabet} $\Sigma$
\item Endliche Menge $V$ mit $V\cap\Sigma=\set{}$ von \tcol{Variablen}
\item \tcol{Startsymbol} $S\in V$
\item Endliche Menge von \tcol{Ableitungsregeln} $R$. Dabei ist eine Ableitungsregel ein Paar 
\[(l,r)\in(V\cup\Sigma)^+\times(V\cup\Sigma)^*.\]
Wir schreiben auch $l\rightarrow r$.
\end{itemize}
$L(G)$ bezeichnet die von $G$ \tcol{erzeugte Sprache} $\set{w\in\Sigma^*\mid S\rightarrow^*w}$.
\end{defi}

\begin{defi}[Chomsky-Hierarchie\index{Chomsky-Hierarchie}]
\begin{description}
\item[Typ 0] Grammatiken ohne weitere Einschränkungen.
\item[Typ 1 oder kontextsensitiv\index{kontextsensitive Sprache}] Grammatiken, bei denen alle Ableitungsregeln die Form habe:
	\begin{itemize}
	\item $u\rightarrow v$ mit $u\in V^+,v\in((V\cup\Sigma)\setminus\set{S})^+$ und $|u|\leq |v|$, oder
	\item $S\rightarrow\epsilon$
	\end{itemize}
\item[Typ 2 oder kontextfrei\index{kontextfrei}] Grammatiken, bei denen alle Ableitungsregeln die Form haben:
	\begin{itemize}
	\item $A\rightarrow v$ mit $A\in V$ und $v\in(V\cup\Sigma)^*$
	\end{itemize}
\item[Typ 3 oder rechtslinear\index{rechtslinear}] Grammatiken, bei denen alle Ableitungsregeln die Form haben:
	\begin{itemize}
	\item $A\rightarrow v$ mit $A\in V$ und $v=\epsilon$ oder $v=aB$ mit $a\in\Sigma,B\in V$
	\end{itemize}
\end{description}
\end{defi}

\begin{satz}
Die Chomsky-Hierarchie\index{Chomsky-Hierarchie} ist echt. Also $L_3\subset L_2\subset L_1\subset L_0$, wobei $L_i$ Klasse der durch Typ-$i$-Grammatiken erzeugten Sprachen.
\end{satz}


\subsection{Chomsky-0-Grammatiken und rekursiv aufzählbare Sprachen}

\begin{satz}
Falls $L$ rekursiv-aufzählbar\index{rekursiv-aufzählbar} (semi-entscheidbar)\index{semi-entscheidbar} ist, so gibt es eine Chomsky-0-Grammatik G mit $L(G)=L$.
\end{satz}

\begin{satz}
Die von Typ-0-Grammatiken $G$ erzeugten Sprachen sind rekursiv-aufzählbar\index{rekursiv-aufzählbar}.
\end{satz}


\subsection{Chomsky-1-Grammatiken bzw. kontextsensitive Sprachen\index{kontextsensitive Sprache}}

\begin{defi}[DTAPE\index{DTAPE} und NTAPE\index{NTAPE}]
$DTAPE(s(n))$ ist die Klasse der Sprachen $L$, für die es eine DTM mit Platzbedarf $s(n)$ (bei Eingabelänge n) gibt, die $L$ akzeptiert.\\
$NTAPE(s(n))$ ist die Klasse der Sprachen $L$, für die es eine NTM mit Platzbedarf $s(n)$ (bei Eingabelänge n) gibt, die $L$ akzeptiert.\\
Es gilt $NTAPE(n)=NTAPE(f(n))$ für eine lineare Funktion $f$. Es ist offen, ob $NTAPE(n)=DTAPE(n)$.
\end{defi}

\begin{satz}
Die Klasse der von Chomsky-1-Grammatiken erzeugten Sprachen stimmt mit der Klasse $NTAPE(n)$ überein.
\end{satz}


\subsection{Chomsky-2-Grammatiken bzw. kontextfreie Sprachen\index{kontextfreie Sprache} und Syntaxbäume}

\begin{defi}[eindeutige Grammatik\index{eindeutige Grammatik}]
Eine kontextfreie Grammatik $G$ heißt \tcol{eindeutig}, wenn es für jedes Wort $w\in L(G)$ \tcol{genau einen Syntaxbaum} gibt. Eine kontextfreie Sprache $L$ heißt eindeutig, wenn es eine eindeutige Grammatik $G$ mit $L(G)=G$ gibt. Ansonsten heißt $L$ \tcol{inhärent mehrdeutig}.\\
Bemerke: Zu jedem Syntaxbaum gehören mehrere Ableitungen zu jeder Ableitung aber nur ein Syntaxbaum.
\end{defi}

\begin{defi}[Chromsky-Normalform\index{Chromsky-Normalform}]
Eine kontextfreie Grammatik ist in Chromsky-Normalform, wenn alle Regeln von der Form:
\begin{itemize}
\item $A\rightarrow BC$ oder
\item $A\rightarrow a$
\end{itemize}
sind, mit $A,B,C\in V$ und $a\in\Sigma$.\\
Für kontextfreie Sprachen, die $\epsilon$ enthalten, lässt sich die Grammatik durch $S'\rightarrow\epsilon$ und $S'\rightarrow S$ ergänzen. Dies wird als \tcol{erweiterte Chomsky-Normalform\index{erweiterte Chomsky-Normalform}} bezeichnet.\\
Jede kontextfreie Grammatik kann in eine Grammatik in erweiterte Chomsky-Normalform\index{erweiterte Chomsky-Normalform} überführt werden.
\end{defi}

\begin{verf}[Umwandlung in Chomsky-Normalform\index{Chomsky-Normalform}]
\begin{description}
\item[Schritt 1:] Ersetze in allen Regeln, bei denen auf der rechten Seite ein Terminalsymbol $a\in\Sigma$ nicht alleine steht durch $Y_a\in V$ und füge die Regel $Y_a\rightarrow a$ hinzu.
\item[Schritt 2:] Ersetze Regeln der Art $A\rightarrow B_1\dots B_m$ mit $m>2$ durch die Regeln.
\begin{align*}
A&\rightarrow B_1C_1\\
C_i&\rightarrow B_{i+1}C_{i+1} \text{ für } 1\leq i\leq m-3\\
C_{m-2}&\rightarrow B_{m-1}B_m
\end{align*}
\item[Schritt 3:] Finde die Menge $V'$ aller Variablen $A$ für die gilt $A\rightarrow^*\epsilon$. Streiche alle Regeln $A\rightarrow\epsilon$. Für $A\rightarrow BC$ wird die Regel $A\rightarrow C$ hinzugefügt, falls $B\in V'$ (analog für $C\in V'$).
\item[Schritt 4:] Zur Ersetzung von Kettenregeln $A\rightarrow B$. Finde zuerst alle Kreise \[A_1\rightarrow A_2\rightarrow\dots\rightarrow A_r\rightarrow A_1\] und ersetzte alle $A_i$ durch $A_1$.\\
Für Regel $A\rightarrow B$ und jede Regeln $B\rightarrow b$ füge Regel $A\rightarrow b$ hinzu und lösche $A\rightarrow B$.
\end{description}
\end{verf}

\begin{satz}[Cocke-Younger-Kasami Algorithmus\index{Cocke-Younger-Kasami Algorithmus} (CYK Algorithmus)]
Das Wortproblem zu kontextfreien Grammatiken in Chomsky-Normalform\index{Chomsky-Normalform} kann in polynomieller Zeit gelöst werden (Abhängig von Wortgröße und Anzahl der Ableitungsschritte).
\end{satz}

\begin{verf}[Cocke-Younger-Kasami Algorithmus\index{Cocke-Younger-Kasami Algorithmus} (CYK Algorithmus)]
Es wird \tcol{dynamische Programmierung} verwendet.\\
Sei $w=w_1\dots w_n$ und $V_{ij}\subseteq V$ die Menge an Variablen $A$, so dass $A\rightarrow^*w_i\dots w_j$ impliziert. Baue eine Tabelle mit wachsendem $l\coloneqq j-i$ auf beginnend mit $l=0$.\\
Für $l=0$ ist $V_{ii}=\set{A\mid a\rightarrow w_i}$.\\
Für $l>0$ muss das zu untersuchende Wort $w_i\dots w_j$ für ein $k$ mit $i\leq k<j$ aufgeteilt werden in $w_i\dots w_k$ und $w_{k+1}\dots w_j$. Anschließend wird untersucht ob eine Ableitungsregeln $A\rightarrow BC$ existiert mit $B\rightarrow^* w_i\dots w_k$ und $C\rightarrow^* w_{k+1}\dots w_j$.
\end{verf}

\begin{satz}[Pumping-Lemma für kontextfreie Sprachen\index{Pumping-Lemma für kontextfreie Sprachen}]
Für jede kontextfreie Sprache $L$ gibt es eine Konstante $n\in\mathbb{N}$, so dass sich jedes Wort $z\in L$ mit $|z|\geq n$ so als $z=uvwxy$ schreiben lässt, dass $|vx|\geq 1,|vwx|\leq n$ und für alle $I\geq 1$ das Wort $uvîwxîy\in L$ ist.
\end{satz}

\begin{satz}[Ogden's Lemma\index{Ogden's Lemma}]
Für jede kontextfreie Sprache $L$ gibt es eine Konstante $n\in\mathbb{N}$, so dass sich jedes Wort $z\in L$ mit $|z|\geq n$ gilt:\\
Wenn wir in $z$ mindestens $n$ Buchstaben markieren, so lässt sich $z$ so als $z=uvwxy$ schreiben, dass von den mindestens $n$ markierten Buchstaben mindestens einer zu $vx$ und höchstens $n$ zu $vwx$ gehören und für alle $I\geq 1$ das Wort $uvîwxîy\in L$ ist.
\end{satz}

\begin{defi}[Nutzlose Variablen\index{nutzlose Variablen}]
Sei $G$ eine kontextfreie Grammatik\index{kontextfreie Grammatik}. Eine Variable $A$ heißt nutzlos, falls es keine Ableitung $S\rightarrow^* w$ gibt, mit $w\in\Sigma^*$, in der $A$ vorkommt.
\end{defi}

\begin{satz}
Für eine kontextfreie Grammatik\index{kontextfreie Grammatik} kann die Menge der \tcol{nutzlosen Variable} in polynomialer Laufzeit berechnet werden.
\end{satz}

\begin{verf}[Bestimmen von nutzlose Variablen\index{nutzlose Variablen}]
Berechne zuerst $V'\subseteq V$, mit $A\in V'$ genau dann, wenn es $w\in\Sigma^*$ gibt mit $A\rightarrow^*w$\\
Berechne anschließend $V''\subseteq V'$ aller $A\in V'$, für die es eie Ableitung $S\rightarrow^*\alpha A\beta$ mit $\alpha\beta\in(V'\cup\Sigma)^*$ oder $A=S$.\\
$V''$ ist die Menge der \tcol{nützlichen Variablen}.
\end{verf}

\begin{satz}
Für eine kontextfreie Grammatik $G$ kann in polynomialer Zeit berechnet werden, ob $L(G)=\set{}$ oder $L(G)$ endlich.
\end{satz}

\begin{satz}
Die Klasse der kontextfreie Sprachen ist \tcol{abgeschlossen} bzgl. Vereinigung, Konkatenation und Kleenschen Abschluss\\
Die Klasse der kontextfreie Sprachen ist \tcol{nicht abgeschlossen} bzgl. Komplementbildung und Durchschnitt\\
\end{satz}

\begin{defi}[Greibach-Normalform\index{Greibach-Normalform}]
Eine kontexfreie Grammatik $G$ ist in Greibach-Normalform, wenn allle Ableitungsregeln der Form
\[A\rightarrow a\alpha\text{ mit }A\in V,a\in\Sigma\text{ und }\alpha\in V^*\]
sind.\\
Jede kontexfreie Grammatik $G$, für die $L(G)$ das leere Wort nicht enthält, kann in eine Greibach-Normalform überführt werden.
\end{defi}

%TODO Verfahren für Greibach-Normalform

\begin{defi}[Kellerautomat\index{Kellerautomat}(NPDA bzw. PDA)]
Ein (nichtdeterministischer) Kellerautomat besteht aus $(Q,\Sigma,\Gamma,q_0,Z_0,\delta,F)$,wobei
\begin{itemize}
\item $Q$ \tcol{endliche Zustandsmenge}
\item $\Sigma$ \tcol{endliches Eingabealphabet}
\item $\Gamma$ \tcol{endliches STACK-Alphabet}
\item $q_0\in Q$ \tcol{Anfangszustand}
\item $Z_0\Gamma$ \tcol{Initialisierung des STACK}
\item $\delta\colon Q\times(\Sigma\cup\set{\epsilon})\times\Gamma\to 2^{Q\times\Gamma^*}$, d.h. $\delta(q,a,Z)\subseteq\set{(q,\gamma)\mid q\in Q,\gamma\in\Gamma^*}$
\end{itemize}
Ein PDA \tcol{akzeptiert} ein $w\in\Sigma^*$ \tcol{durch leeren Stack}, wenn es eine zulässige Folge von Konfigurationen aus der Anfangskonfiguration $(q_0,w,Z_0)$ in eine Endkonfiguration $(q,\epsilon,\epsilon)$ gibt.\\ 
Ein PDA \tcol{akzeptiert} ein $w\in\Sigma^*$ \tcol{durch einen akzeptierenden Endzustand}, wenn es eine zulässige Folge von Konfigurationen aus der Anfangskonfiguration $(q_0,w,Z_0)$ in eine Konfiguration $(q,\epsilon,\gamma)$ mit $q\in F$ und $\gamma\in\Gamma^*$ gibt.
\end{defi}

\begin{satz}[Äquivalenz des Akzeptanzverhalten von PDAs]
Zu einem PDA\index{Kellerautomat}, der eine Sprache $L$ durch einen akzeptierenden Endzustand akzeptiert, kann ein PDA konstruiert werden, der $L$ mit leerem STACK akzeptiert \tcol{und umgekehrt}.
\end{satz}

\begin{satz}
Die klasse der von nichtdeterministischen Kellerautomaten akzeptierten Sprachen ist gleich der Klasse der kontextfreien Sprachen.
\end{satz}


\subsection{Chomsky-3-Grammatiken und reguläre Sprachen\index{reguläre Sprachen}}

\begin{satz}
Die Klasse der von endlichen Automaten\index{endlicher Automat} akzeptieren Sprachen ist genau die Klasse von Chomsky-3-Grammatiken erzeugten Sprachen
\end{satz}


\subsection{Unentscheidbare Probleme für kontextfreie Grammatiken\index{kontextfreie Grammatiken}}

\begin{satz}
Das Problem für kontextfreie Grammatiken $G_1$ und $G_2$ zu entscheiden, ob $L(G_1\cap G_2=\set{}$ ist, ist \tcol{nicht entscheidbar}.
\end{satz}

\begin{satz}
Das Problem für eine kontextfreie Grammatiken $G_1$ zu entscheiden, ob sie \tcol{eindeutig} ist, ist \tcol{nicht entscheidbar}.
\end{satz}

\begin{defi}[Sprache der korrekten Rechenwege einer TM]
Die Sprache $B_M$ der korrekten Rechenwege einer TM $M$ besteht aus allen Worten
\[w_1\#w_2^R\#w_3\#w_4^R\dots w_n^R\#\text{, falls $n$ gerade und}\]
\[w_1\#w_2^R\#w_3\#w_4^R\dots w_n\#\text{, falls $n$ ungerade}\]
wobei die $w_i$ \tcol{aufeinanderfolgende Konfigurationen} von $M$ sind.
\end{defi}

\begin{satz}
Für alle Turingmaschinen $M$ ist $B_M$ der Durchschnitt zweier Sprachen $L_1=L(G_1)$ und $L_2=L(G_2)$, wobei $G_1$ und $G_2$ kontextfreie Grammatiken sind.
\end{satz}


\subsection{Zusammenfassung}

\begin{table}[h]
	\centering
	\begin{tabular}[c]{|l|c|c|c|c|c|}
	\hline
	Typ & $w\in L(G)$ & $L(G)=\set{}$ & $L(G)$ endlich & $L(G_1)=L(G_2)$ & $L(G_1)\cap L(G_2)=\set{}$ \\
	\hline
	\hline
	0 & \xmark & \xmark & \xmark & \xmark & \xmark\\
	1 & \cmark & \xmark & \xmark & \xmark & \xmark\\
	2 & \cmark & \cmark & \cmark & \xmark & \xmark\\
	3 & \cmark & \cmark & \cmark & \cmark & \cmark\\
	\hline
	\end{tabular}
\caption{Entscheidbarkeit für Grammatiken der Chomsky-Hierarchie}
\end{table}
\begin{table}[h]
	\centering
	\begin{tabular}[c]{|l|c|c|c|c|c|}
	\hline
	Typ & $L(G)^c$ & $L(G_1)\cdot L(G_2)$ & $L(G_1)\cup L(G_2)$ & $L(G_1)\cap L(G_2)$ & $L(G)^*$\\
	\hline
	\hline
	0 & \xmark & \cmark & \cmark & \cmark & \cmark\\
	1 & \cmark & \cmark & \cmark & \cmark & \cmark\\
	2 & \xmark & \cmark & \cmark & \xmark & \cmark\\
	3 & \cmark & \cmark & \cmark & \cmark & \cmark\\
	\hline
	\end{tabular}
\caption{Abgeschlossenheit für Grammatiken der Chomsky-Hierarchie}
\end{table}



\section{Informationstheorie\index{Informationstheorie}}


\subsection{Quellkodierung\index{Quellkodierung}}

Wir betrachten $\Sigma=\set{1,\dots,n}$ eine Menge von Zeichen mit Wahrscheinlichkeit $\set{p_1,\dots,p_n}$. $X$ (Zufallsvariable/Informationsquelle) liefert $i\in\Sigma$ mit Wahrscheinlichkeit $p_i$.

\begin{defi}[Information\index{Information}]
Die Information einer Wahrscheinlichkeit $p$ ist
\[I\coloneqq\log_2\left(\frac{1}{p}\right)=-\log_2(p).\]
\end{defi}

\begin{defi}[Entropie\index{Entropie}]
Die Entropie einer diskreten Zufallsvariable $X$ mit Wahrscheinlichkeiten $p(x)$ für $x\in X$ ist definiert durch
\[H(X)\coloneqq\sum_{x\in X}p(x)\log_2\left(\frac{1}{p(x)}\right)\]
\end{defi}

%TODO Hier fehlt noch viel aber ich fand das auch nicht so wichtig..



\section{Probleme\index{Problem}}


\subsection{Definitionen und Begriffe}

\begin{defi}[Problemvarianten]
Zu den meisten Problemen gibt es verschiedene Problemvarianten.
\begin{itemize}
\item \tcol{Optimierungsproblem}\index{Optimierungsproblem}: Gesucht ist die optimale Lösung eines gegebenen Problems.
\item \tcol{Optimalwertproblem}\index{Optimalwertproblem}: Gesucht ist der Wert der optimalen Lösung.
\item \tcol{Entscheidungsproblem}\index{Entscheidungsproblem}: Es soll entschieden Werden, ob es Lösungen gibt, die besser als ein gegebener Wert $k$ sind.
\end{itemize}
\end{defi}

\begin{defi}[Kodierungsschema\index{Kodierungsschema}]
Ein \tcol{Kodierungsschema} ordnet jeder Probleminstanz eines Problems eine Zeichenkette über dem Alphabet $\Sigma$ zu. Die \tcol{Inputlänge}\index{Inputlänge} eines Problembeispiels ist die Anzahl der Symbole der Kodierung.
\end{defi}

\begin{defi}[äquivalente Kodierungsschemata\index{äquivalente Kodierungsschema}]
Zwei Kodierungsschemata heißen \tcol{äquivalent} bezüglich eines Problems $\Pi$, wenn sie sich gegenseitig durch Polynome abschätzen lassen für alle Instanzen von $\Pi$.
\end{defi}

\begin{defi}
Zu einem Entscheidungsproblem $\Pi$ bezeichnen wird die Menge an Probleminstanzen als $D_\Pi$, die Teilmenge der \tcol{Ja-Instanzen} mit $J_\Pi\subseteq D_\Pi$ und der \tcol{Nein-Instanzen} mit $N_\Pi\subseteq D_\Pi$.
\end{defi}

\begin{defi}
Die zu einem Problem $\Pi$ und einem Kodierungsschema $s$ \tcol{zugehörige Sprache} ist
\[L[\Pi,s]\coloneqq\set{x\in\Sigma^*\mid x\text{ ist Kodierung einer Ja-Instanz von $\Pi$ unter $s$}}\]
Eine deterministische Turingmaschine $M$ \tcol{löst} ein Entscheidungsproblem $\Pi$ unter der Kodierung $s$, wenn $M$ für jede Eingabe hält und $L_M=L[\Pi,s]$.
\end{defi}

\begin{defi}[Suchproblem\index{Suchproblem}]
Ein Suchproblem $\Pi$ wird beschrieben durch eine Menge von Probleminstanzen $D_\Pi$ und für $I\in D_\Pi$ die Menge $S_\Pi(I)$ \tcol{aller} Lösungen von $I$.\\
Die Lösung eines Suchproblems besteht in der \tcol{Angabe einer Lösung} aus $S_\Pi(I)$.
\end{defi}

\begin{defi}[Aufzählungsproblem\index{Aufzählungsproblem}]
Ein Aufzählungsproblem $\Pi$ wird beschrieben durch eine Menge von Probleminstanzen $D_\Pi$ und für $I\in D_\Pi$ die Menge $S_\Pi(I)$ \tcol{aller} Lösungen von $I$.\\
Die Lösung eines Aufzählungsproblem besteht in der \tcol{Angabe der Kardinalität} von $S_\Pi(I)$ also $|S_\Pi(I)|$.
\end{defi}

\begin{defi}[Reduzierbarkeit für Suchprobleme]
Zu einem Suchproblem $\Pi$ sei die Relation $R_\Pi$ gegeben:
\[R_\Pi\coloneqq\set{(x,c)\mid x\in D_\Pi,s\in S_\Pi(x)}\]
Eine Funktion $f\colon\Sigma^*\to\Sigma^*$ \tcol{realisiert} eine Relation $R$, wenn für alle $x\in\Sigma^*$ gilt:
\[f(x)= \begin{cases}
		\epsilon & \nexists y\in\Sigma^*\setminus\set{\epsilon}\colon (x,y)\in R \\
		y & \text{sonst, mit beliebigem } y\colon (x,y)\in R
		\end{cases}
\]
Ein Algorithmus \tcol{löst} das durch $R_\Pi$ beschriebene Suchproblem $\Pi$, wenn er eine Funktion berechnet, die $R_\Pi$ berechnet.
\end{defi}


\subsection{Beispiele}

\begin{prob}[Post'sche Korrespondenzproblem\index{Post'sche Korrespondenzproblem}]
Gegeben ist eine endliche Folge von Wortpaaren
\[K=((x_1,y_1),\dots,(x_n,y_n))\]
über einem endlichen Alphabet $\Sigma$. Es gilt $x_i\neq\epsilon$ und $y_i\neq\epsilon$.\\
Gesucht ist, eine endliche Folge von Indizes $i_1,\dots,i_k\in\set{1,\dots,n}$, so dass $x_{i_1}\dots x_{i_k}=y_{i_1}\dots y_{i_k}$ gilt.\\
Dieses Problem ist nicht entscheidbar\index{Unentscheidbarkeit}.
\end{prob}

\begin{prob}[Traveling Salesman Problem\index{Traveling Salesman Problem} (TSP)]
Gegeben sei ein vollständiger Graph $G=(V,E)$, sowie eine Kostenfunktion $c\colon E\to\mathbb{Z}^+$.\\
Gesucht ist eine Tour, die alle Elemente aus $V$ enthält und minimale Gesamtlänge hat.\\
Das TSP ist $NP$-vollständig\index{NP-vollständig}\\
Für das TSP mit \tcol{Dreiecksungleichung} existiert ein Approximationsalgorithmus $A$ mit $R_A\leq 2$ für alle Instanzen $I$.
\end{prob}

\begin{prob}[SAT\index{SAT}]
Gegeben ist eine Menge $U$ von Variablen, und eine Menge $C$ von Klauseln über $U$.\\
Gesucht ist eine Wahrheitsbelegung von $U$, so dass $C$ erfüllt wird. Also in jeder Klausel mindestens eine Variable Wahr ist.\\
SAT ist ist $NP$-vollständig\index{NP-vollständig}.
\end{prob}

\begin{prob}[3SAT\index{3SAT}]
Gegeben ist eine Menge $U$ von Variablen, und eine Menge $C$ von Klauseln über $U$, wobei jede Klausel \tcol{genau drei} Literale enthält.\\
Gesucht ist eine Wahrheitsbelegung von $U$, so dass $C$ erfüllt wird. Also in jeder Klausel mindestens eine Variable Wahr ist.\\
3SAT ist ist $NP$-vollständig\index{NP-vollständig}.
\end{prob}

\begin{prob}[2SAT\index{2SAT}]
Gegeben ist eine Menge $U$ von Variablen, und eine Menge $C$ von Klauseln über $U$, wobei jede Klausel \tcol{genau zwei} Literale enthält.\\
Gesucht ist eine Wahrheitsbelegung von $U$, so dass $C$ erfüllt wird. Also in jeder Klausel mindestens eine Variable Wahr ist.\\
Im Gegensatz zu 3SAT ist 2SAT$\in P$.
\end{prob}

\begin{prob}[Max2SAT\index{Max2SAT}]
Gegeben ist eine Menge $U$ von Variablen, und eine Menge $C$ von Klauseln über $U$, wobei jede Klausel \tcol{genau zwei} Literale enthält und eine Zahl $K\in\mathbb{N}$\\
Gesucht ist eine Wahrheitsbelegung von $U$, so dass \tcol{mindestens $K$ Klauseln} aus $C$ erfüllt wird.\\
Max2SAT ist $NP$-vollständig\index{NP-vollständig}.
\end{prob}

\begin{prob}[CLIQUE\index{CLIQUE}]
Gegeben ist ein Graph $G=(V,E)$ und ein Parameter $K\leq |V|$\\
Gesucht ist eine \tcol{Clique} $V'\subseteq V$ der Größe $K$, sodass für alle $i,j\in V', i\neq j$ gilt: $\set{i,j}\in E.$
CLIQUE ist $NP$-vollständig\index{NP-vollständig}.\\
Falls $P\neq NP$ existiert kein absoluter Approximationsalgorithmus\index{absoluter Approximationsalgorithmus} für CLIQUE.
\end{prob}

\begin{prob}[COLOR\index{COLOR}]
Gegeben ist ein Graph $G=(V,E)$ und ein Parameter $K\in\mathbb{N}$\\
Gesucht ist eine \tcol{Knotenfärbung} von $G$ mit höchstens $K$ Farben, so dass je zwei adjazente Knoten verschiedene Farben besitzen.\\
3COLOR bezeichnet das Problem COLOR mit festem $K=3$. 3COLOR ist $NP$-vollständig\index{NP-vollständig}.\\
Falls $P\neq NP$, dann existiert kein relativer Approximationsalgorithmus\index{relativer Approximationsalgorithmus} $A$ für COLOR mit $R^{\,\infty}_A<\frac{4}{3}$.
\end{prob}

\begin{prob}[SUBSET SUM\index{SUBSET SUM}]
Gegeben ist eine endliche Menge $M$, eine Gewichtsfunktion $w\colon M\to\mathbb{N}_0$ und $K\in\mathbb{N}_0$.\\
Gesucht ist eine Teilmenge $M'\subseteq M$ mit
\[\sum_{a\in M'}w(a)=K.\]
SUBSET SUM ist $NP$-vollständig\index{NP-vollständig}.
\end{prob}

\begin{prob}[PARTITION\index{PARTITION}]
Gegeben ist eine endliche Menge $M$ und eine Gewichtsfunktion $w\colon M\to\mathbb{N}
_0$.\\
Gesucht ist eine Teilmenge $M'\subseteq M$ mit
\[\sum_{a\in M'}w(a)=\sum_{a\in M\setminus M'}w(a)\]
PARTITION ist $NP$-vollständig\index{NP-vollständig}.
\end{prob}

\begin{prob}[KNAPSACK\index{KNAPSACK}]
Gegeben ist eine endliche Menge $M$, eine \tcol{Gewichtsfunktion} $w\colon M\to\mathbb{N}$ und eine \tcol{Kostenfunktion} $c\colon M\to\mathbb{N}
_0$ und $W,C\in\mathbb{N}_0$.\\
Gesucht ist eine Teilmenge $M'\subseteq M$ mit
\[\sum_{a\in M'}w(a)\leq W\text{ und }\sum_{a\in M'}c(a)\geq C\]
KNAPSACK ist $NP$-vollständig\index{NP-vollständig}.\\
Falls $P\neq NP$ existiert kein absoluter Approximationsalgorithmus\index{absoluter Approximationsalgorithmus} für KNAPSACK. Es gibt einen einfachen \tcol{Greedy-Algorithmus} $A$, mit $R_A(I)\leq 2$ für alle Instanzen $I$.\\
Für KNAPSACK existiert ein FPAS. Es verwendet \tcol{dynamische Programmierung mit skalierten Eingabegrößen}.
\end{prob}

\begin{prob}[Hamilton-Kreis\index{Hamilton-Kreis}]
Gegeben ist ein ungerichteter, ungewichteter Graph $G=(V,E)$.\\
Gesucht ist ein \tcol{Hamilton-Kreis} in $G$. Ein Hamilton-Kreis ist eine Permutation $\pi$ auf $V$, so dass $\set{\pi(n),\pi(1)}\in E$ und $\set{\pi(i),\pi(i-1)}\in E$ für $1\leq i\leq n-1$.
\end{prob}

\begin{prob}[Subgraphisomorphie\index{Subgraphisomorphie}]
Gegeben sind Graphen $G$ und $H$ wobei \tcol{$H$ weniger Knoten als $G$} besitzt.\\
Es wird danach gefragt, ob $H$ isomorph zu einem Subgraphen von $G$ ist.\\
Subgraphisomorphie ist $NP$-vollständig\index{NP-vollständig}.
\end{prob}

\begin{prob}[Graphisomorphie\index{Graphisomorphie}]
Gegeben sind Graphen $G$ und $H$ mit \tcol{gleich vielen Knoten}.\\
Es wird danach gefragt, ob $G$ und $H$ zueinander isomorph sind.\\
Graphisomorphie ist ein Kandidat für ein Problem aus $NPI$\index{NPI}. Vielleicht ist es auch in $P$\index{P}.
\end{prob}

\newpage
\glqq Da ich den kürzesten Weg durch die Vorlesung gefunden habe, sind wir jetzt schon fertig.\grqq{}

\printindex

\end{document}
